\chapter{Proposed Study Setting}

\section{Hypothesises}
\section{Variables}
\subsection{Independent Variables}
\begin{itemize}
	\item ego-centric
	\item exo-centric
	\item combined
	\begin{itemize}
		\item \textcolor{red}{variation of combined:} sequetial - parallel
	\end{itemize}
\end{itemize}
 further variations that could be interesting:
 \begin{itemize}
	\item body parts
	\item sitting/standing
	\item visual representation
	\item degree of realism of avatar
	\item guidance techniques: stopping at keyframes vs. fluent instructions
	\item audio queues
	\item feedback
	\item fixed position of teacher and avatar, vs walking around
\end{itemize}
\subsection{Dependent variables}
Score. Combination of objective and subjective measurements.
\begin{itemize}
	\item precision
	\item subjective opinion of participant
	\item stress level (EKG, HRV)
	\item cognitive load
	\item retention
	\item reaction time
\end{itemize}
\section{Task}
	\begin{itemize}
		\item Tai Chi form, split in subtasks
		\item Dance moves from single dance
	\end{itemize}
variations:
\begin{itemize}
	\item difficulty
	\item complexity
	\item abstract vs. real world
	\item operate a control panel
	\item escape the room
	\item game	
\end{itemize}
\begin{comment}

\section{Hypothesis}
%hier wird ein mögliches studiendesign vorgestellt

\subsection{Aim of the Study}
The aim of the study is to investigate the influence of egocentric and exocentric perspectives on a virtual avatar during motor learning tasks.

\subsection{process}
There are two groups: one learn only with the egocentric perspective, the other one with the exocentric perspective on the virtual avatar.\\
To derive conclusions on body regions, every participant learns movements for three different body parts. The body parts are:
\begin{itemize}
	\item \UB (UB)
	\item \LB (LB)
	\item \FB (FB)
\end{itemize}
To derive conclusion on movement types, two different movements per body part is learned. The two movement types are:
\begin{itemize}
	\item mirrored movements
	\item independent movements
\end{itemize}

\begin{center}
	\begin{tabular}{ | c | c | c | c | }
		\hline
		& UB & LB & FB \\ \hline 
		Ego & \parbox{4cm}{1 mirrored and 1 asynchronous movement} & \parbox{4cm}{1 mirrored and 1 independent movement} & \parbox{4cm}{1 mirrored and 1 independent movement} \\ \hline 
		Exo & \parbox{4cm}{1 mirrored and 1 independent movement} & \parbox{4cm}{1 mirrored and 1 independent movement} & \parbox{4cm}{1 mirrored and 1 independent movement} \\ \hline
		Ego/Exo & \parbox{4cm}{1 mirrored and 1 independent movement} & \parbox{4cm}{1 mirrored and 1 independent movement} & \parbox{4cm}{1 mirrored and 1 independent movement} \\
		\hline
	\end{tabular}
	\label{table:studyDesign}
\end{center}

\subsection{Independent variables}
\begin{itemize}
	\item perspective on the avatar (Ego/Exo centric)
	\item body parts (\UB, \LB, \FB)
	\item movement types (mirrored/independent movements)
\end{itemize}

\subsection{measures}
TBA
\end{comment}