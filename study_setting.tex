\chapter{Study Setting/ concept}

\section{Preliminary Study design}
\subsection{Aim of the Study}
The aim of the study is to investigate the influence of egocentric and exocentric perspectives on a virtual avatar during motor learning tasks.

\subsection{process}
There are two groups: one learn only with the egocentric perspective, the other one with the exocentric perspective on the virtual avatar.\\
To derive conclusions on body regions, every participant learns movements for three different body parts. The body parts are:
\begin{itemize}
	\item \UB (UB)
	\item \LB (LB)
	\item \FB (FB)
\end{itemize}
To derive conclusion on movement types, two different movements per body part is learned. The two movement types are:
\begin{itemize}
	\item mirrored movements
	\item independent movements
\end{itemize}

\begin{center}
	\begin{tabular}{ | c | c | c | c | }
		\hline
		 & UB & LB & FB \\ \hline 
		Ego & \parbox{4cm}{1 mirrored and 1 asynchronous movement} & \parbox{4cm}{1 mirrored and 1 independent movement} & \parbox{4cm}{1 mirrored and 1 independent movement} \\ \hline 
		Exo & \parbox{4cm}{1 mirrored and 1 independent movement} & \parbox{4cm}{1 mirrored and 1 independent movement} & \parbox{4cm}{1 mirrored and 1 independent movement} \\ \hline
		Ego/Exo & \parbox{4cm}{1 mirrored and 1 independent movement} & \parbox{4cm}{1 mirrored and 1 independent movement} & \parbox{4cm}{1 mirrored and 1 independent movement} \\
		\hline
	\end{tabular}
\end{center}

\subsection{Independent variables}
\begin{itemize}
	\item perspective on the avatar (Ego/Exo centric)
	\item body parts (\UB, \LB, \FB)
	\item movement types (mirrored/independent movements)
\end{itemize}

\subsection{measures}
TBA


\subsection{comments}
um \underline{Influence of Egocentric and Exocentric perspectives on Virtual Avatars during full-body and part-body mirrored and independent motor learning tasks} zu zeigen, muss ich dann nicht auch verschiedene lern systeme heranziehen? sonst zeige ich das ja nur für dieses eine lern system.
