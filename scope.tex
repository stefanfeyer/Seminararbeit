\chapter{Scope}
\section{Motor Learning}
beschreibung welche auf welche art von bewegungen sich hier beschränkt wird.

\section{Mixed Reality}
begründen welchen auf welchen part des kontinuums man sich beschränkt

\section{Perspective}
welche prespectiven behandelt werden und welche nicht


\section{Misc}
welche einschränkungen gibt es noch. zb. das es kein feedback gibt, der realitätsgrad der avatare nicht evaluiert werden soll. erst zum schluss


\begin{comment}
\begin{itemize}
	\item discrete movements
	\item closed skills
	\item at least 2 different movement categories
	\item how to measure movements
\end{itemize}
\section{Mixed Reality}
\begin{itemize}
	\item Milgram
	\item AR or VR
\end{itemize}
\section{other aspects}
\begin{itemize}
	\item synchron asynchron
	\item colocated/remote
	\item perspective
	\item hardware?
	\item feedback!
	\item real world, not abstract avatars
	\item only visuals - no audio or textual explanation
\end{itemize}

	\begin{itemize}
		\item 26: details following: how to measure movements for movements with a discrete target
		\item 3 types of measurements: measures of error for a single subject, measures of time and speed, measures of movement magnitude.
		\begin{itemize}
		\item \underline{Constant Error}: average Error $CE=\frac{\sum(x_i-T)}{n}$. i: all values, T: target value, n: number of values. interpretation: in average, the user missed the target by CE
		\item \underline{Variable Error}: inconcistency in movement error: $VE=\sqrt{\frac{\sum(x_i-M)^2}{n}}$. M: average movement, actual movement score - average movement score. interpretation: VE reflects the variability, or inconsistency in movements. moves consistently: VE small. user moves absolute consistently: VE is 0. VE does not depend on wether or not the subject was close to the target
		\item \underline{total variability}: the total variability around a target: $E=VE^2+CE^2=\sqrt{\frac{\sum(x_i-T)^2}{n}}$
		interpretation: combination of VE and CE, total amount of spread about the target: overall measure how successful was the subject in achieving the target
		\item \underline{absolute error}: measure of overall accuracy in performance. $AE=\frac{\sum|x_i-T|}{n}$. interpretation: replace sqrt with abs
		\item \underline{AE vs. E}: \todo
		\item \underline{Absolute Constant Error}: $=|CE|$. if half pos and half neg could cancel each other out. when mean.
		\item these measures can be applied to other movements. like pursuit motor: TOT, Mashburn task, stabilometer, two hand coordination task.
	\end{itemize}
		
		\item \underline{measures of time and speed}: basic to this idea: performer who can accomplish more in a given amount of time or who can accomplish a given amount of behavior is  more skillfull. time measure:c $\frac{time}{unit}$. speed:$\frac{units}{time}$.
		\item \underline{reaction time} (RT): can also be a performance measure. a measure of time from the arrival of a sudden and unanticipated signal to the beginning of the response. 
		i will only describe it if i will use it
		\item \underline{movement time (MT):} how long does the movement last. somtimes commbined with RT: response time$=RT+MT$
		
	\end{itemize}
	\begin{itemize}
		\item 21 details following: discrete/closed skills
		\item for simplifying discussion introducing classification of movements and motor tasks.
		\item 2 important classification schemes:
		\begin{itemize}
			\item based on particular movements made: discrete, continuous, serial
			\item based on perceptual attributes of the task: open/closed skills
		\end{itemize}
		\item \underline{discrete movements:} movements with recognisable beginning and end. discrete tasks: kicking a ball, shifting gears. end of movement: the time on which a observer ceased examining. dm can be very rapid like blinking or longer like making the signing.
		\item \underline{continuous movements:} dont have recognisable start and end, with behavior continuing till the movement arbitrarily stopped. Continuous tasks: swimming, running, steering a car. Continuous tasks tend to be longer than discrete tasks.
		\item \underline{serial movements:} neither discrete nor continuous compromised of a series of individual movements tied together in time to make some "whole". center of continuum. can be rather long but are not stopped arbitrarily. serial tasks: starting a car, prepareing and lighting a wood fireplace. Serial tasks can be seen as many discrete tasks strung together and the order (and sometimes timing) is important.
		\item \underline{open skills:} environment is constantly, unpredictably changing, so the performer cannot plan his activity effectively in advance. eg. penalty shot in ice hockey. own movement is dependet on the movement of the keeper. Driving on a freeway: depends on the other cars. Success in open skills largely determined by the extend to which a individual can adapt the planned motor behaviour to the changing environment.
		\item \underline{closed skills:} other end of continuum, predictable environment becaus it is stable. eg archery, bowling or signing. movement can be planned in advance.
		since open skills  seems to require rapid adaptions to a changing environment and closed skills require a very stable performances in a predictable environment questions are raised about the method of training, do different individuals perform better in in one of these skill classes. 
		\item to overcome these question the focus of this seminar is on discrete movement tasks and closed skills. §->§ see stdudy
	\end{itemize}
\end{comment}