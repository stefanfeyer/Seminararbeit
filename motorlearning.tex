\section{Motor Learning}
\begin{itemize}
	\item 21
	\item for simplifying discussion introducing classification of movements and motor tasks.
	\item 2 important classification schemes:
	\begin{itemize}
		\item based on particular movements made: discrete, continuous, serial
		\item based on perceptual attributes of the task: open/closed skills
	\end{itemize}
	\item \underline{discrete movements:} movements with recognisable beginning and end. discrete tasks: kicking a ball, shifting gears. end of movement: the time on which a observer ceased examining. dm can be very rapid like blinking or longer like making the signing.
	\item \underline{continuous movements:} dont have recognisable start and end, with behavior continuing till the movement arbitrarily stopped. Continuous tasks: swimming, running, steering a car. Continuous tasks tend to be longer than discrete tasks.
	\item \underline{serial movements:} neither discrete nor continuous compromised of a series of individual movements tied together in time to make some "whole". center of continuum. can be rather long but are not stopped arbitrarily. serial tasks: starting a car, prepareing and lighting a wood fireplace. Serial tasks can be seen as many discrete tasks strung together and the order (and sometimes timing) is important.
	\item \underline{open skills:} enviropnment is constantly, unpredictively changing, so the performer connot plan his activity effectively in advance. eg. penalty shot in ice hockey. own movement is dependet on the movement of the keeper. Driving on a freeway: depends on the other cars. Success in open skills largely determined by the extend to which a individual can adapt the planned motor behaviour  to th echanging environment.
	\item \underline{closed skills:} other end of continuum, predictable environment becaus it is stable. eg archery, bowling or signing. movement can be planned in advance.
	since open skills  seems to require rapid adaptions to a changing environment and closed skills require a very stable performances in a predictable environment questions are raised about the method of training, do different individuals perdorm better in in one of these skill classes. to overcome these question the focus of this seminar is on discrete movement tasks and clsoed skills. -> see stdudy
	
	
\end{itemize}
