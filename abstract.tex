%\section*{Abstract}
\chapter*{Abstract}
Motor learning in the real world takes needs traditionally at least a teacher and a student. The teacher performs a movement, the student mimics this movement and gains proficiency in this task. Motor learning is also present in the digital world, where students can e.g. learn from videos. In these scenarios, the student sees the teacher always in the exo-centric perspective. But with the use of mixed reality technologies, the teacher can stand inside the student's body, allowing the student to see the teacher in the ego-centric perspective. But what are the implications for learning with this change of visual perspective? This work lies theoretical foundations, sets the scope and parameters for a study that aims to investigate these implications, proposing a study design itself.

