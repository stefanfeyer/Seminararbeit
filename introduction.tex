\chapter{Introduction}

\section{Motivation}
%heranführung zum thema, erklären warum das wichtig ist.
In recent years, MR devices became more affordable \footnote{\todo}, portable \footnote{\todo} and usable in many conditions. Not only in academic researchers are interested in this technology, commercial companies also found this technology helpful to explore new possibilities to use it profitable. EON \footnote{\todo} for example calls themself "the world leader in Virtual Reality based knowledge transfer for industry, education, and edutainment". They develop MR programs for several platforms, eg. with the aim to guide workers, reducing mistakes and thus reducing costs.\\
Since MR learning or guiding programs reached the commercial market, many applications will be created. It is important to build these applications on well founded research.\\ 
Developing a system for MR learning can be complex and mistakes can be made. Providing a developer with guidelines to design such a program could help decreasing design faults. But before guidelines can be created, groundwork has to be done and be investigated with sophisticated research methods.
This seminar thesis will take a look in the background of motor learning and perspectives to conduct groundwork that maybe later can be used for guidelines for designing a MR motor learning system.
\begin{itemize}
	\item Overall aim of the Masters theses: provide insights to learning in vr, especially about the perspective on the avatar who is teaching (egocentric vs. exocentric perspective)	
	\item thus: provide groundwork for motor learning in vr for future HCI related studies
\end{itemize}

\section{Problem definition and RQ}
%welche lücken gibt es in der aktuellen forschung
\begin{itemize}
	\item Motor learning tasks can be learn in MR (quellen)
	\item investigations in xyz but not in terms of perspective
	\item influence of perspective could lead to insights/ recommendations for learning in MR
\end{itemize}
Overall rq
\begin{itemize}
	\item How does perspectives on virtual avatars influence motor learning?
\end{itemize}
\begin{comment}
\begin{itemize}
	\item RQ1: Does the perspective on a Virtual Avatar influence the learning performance (?better: outcome?)?
	\item RQ2: When the movement is only on a specific body part like upper body (UB), lower body (LB) or full body (FB), is there a relation between the egocentric or exocentric perspective on the avatar to the learning performance?
	\item H1: The perspective on the avatar has no influence on UB movements
	\item H2: The perspective on the avatar has no influence on LB movements
	\item H3: The perspective on the avatar has no influence on FB movements
	\item H4: The perspective on the avatar has no preferences on movement types, means the movement type has no influence on the learning performance 
\end{itemize}
\end{comment}


\section{Approach}
%heransgehensweise, wie diese lücke gefüllt werden könnte, how to adress the problem
\begin{itemize}
	\item Design a Study, participants to perform movements
	\item two groups, ego/exo perspective, 2 movement types
	\item investigate the performace of the groups
\end{itemize}

\section{outline}
%Übersicht über die arbeit geben
After this introduction, the scope of this thesis is given, where it is explained to what extend motor learning, MR, perspectives and other factors are considered. The following related work part will give an overview about other MR learning systems and also work about perspectives on avatars. From this work the measures, dependent and independent variables and tasks are derived. Taking the related work into consideration a study design is proposed in outlook section.

