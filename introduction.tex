\chapter{Introduction}

\section{Motivation}
%heranführung zum thema, erklären warum das wichtig ist.
In recent years, MR devices became more affordable \footnote{\todo}, portable \footnote{\todo} and usable in many conditions. Not only in academic researchers are interested in this technology, commercial companies also found this technology helpful to explore new possibilites to use it profitable. EON \footnote{\todo} for example calles themself "the world leader in Virtual Reality based knowledge transfer for industry, education, and edutainment". They develop MR programs for several platforms, eg. with the aim to guide workers, reducing mistakes and thus reducing costs.\\
Since MR learning or guiding programs reached the commercial market and many applications exists, it is important to build the applications on well founded research.\\
For example, a developer wants to develop a system for motor learning. 
When developing a system for motor learning it would be great if the developer has the possibility to follow guidelines to design the system. For designing such a
For example there is a system teaching motor movements
aim
This seminar theses will focus on motor learning in MR, especially how the perspective on a virtual avatar influence motor learning in MR.


\section{How do we learn movements}
Beschreiben wie wir generell bewegungen erlernen nicht unbedingt mit bezug auf lernen in MR

\section{Movement types}
gespiegelte, ungespiegelt, synchron, asynchron

\section{How to quantify movements}
Wie können bewegungen überhaupt quantifiziert werden

\section{How to measure movements}
Welche messmethoden gibt es um bewegungen zu messen

\section{Perspectives}
welche perspektieven gibt es

\section{Problem definition and RQ}
welche lücken gibt es in der aktuellen forschung

\section{Approach}
heransgehensweise, wie diese lücke gefüllt werden könnte

\section{outline}
Übersicht über die arbeit geben

\begin{comment}
	

\begin{itemize}
	\item Overall aim of the Masters theses: provide insights to learning in vr, especially about the perspective on the avatar who is teaching (egocentric vs. exocentric perspective)	
	\item thus: provide groundwork for motor learning in vr for future HCI related studies
\end{itemize}
\subsection{Problem definition}
\begin{itemize}
	\item Motor learning tasks can be learn in MR (quellen)
	\item investigations in xyz but not in terms of perspective
	\item influence of perspective could lead to insights/ recommendations for learning in MR
\end{itemize}
\section{Approach}
How to address the Problem
\begin{itemize}
	\item Design a Study, participants to perform movements
	\item two groups, ego/exo perspective, 2 movement types
	\item investigate the performace of the groups
\end{itemize}
\subsection{Overall research question}
\begin{itemize}
	\item How does perspectives on virtual avatars influence motor learning?
\end{itemize}
	
\section{Outline}
After this introduction, the scope of this thesis is given. The Motor Learning movements are described as well as the classification for the Mixed Reality. In the Therory section a classification of this work in relation to the Methodology and HCI Theory. The related work part will give an overview about other and MR learning systems and also work about perspectives on avatars. From this work the measures, dependent and independent variables and tasks are derived. Taking the related work into consideration a study design is proposed in the Study Setting section. Furthermore a outlook is given in the last section.


\end{comment}