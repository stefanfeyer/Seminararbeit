\chapter{Introduction}
\section{Motivation}
\begin{itemize}
	\item Overall aim of the Masters theses: provide insights to learning in vr, especially about the perspective on the avatar who is teaching (egocentric vs. exocentric perspective)	
	\item thus: provide groundwork for motor learning in vr for future HCI related studies
\end{itemize}
\subsection{Problem definition}
\begin{itemize}
	\item Motor learning tasks can be learn in MR (quellen)
	\item investigations in xyz but not in terms of perspective
	\item influence of perspective could lead to insights/ recommendations for learning in MR
\end{itemize}
\section{Approach}
How to address the Problem
\begin{itemize}
	\item Design a Study, participants to perform movements
	\item three groups, ego/exo perspective
	\item investigate the performace of the groups
\end{itemize}
\subsection{Research questions and hypothesis}
\begin{itemize}
	\item RQ1: Does the perspective on a Virtual Avatar influence the learning performance (?better: outcome?)?
	\item RQ2: When the movement is only on a specific body part like upper body (UB), lower body (LB) or full body (FB), is there a relation between the egocentric or exocentric perspective on the avatar to the learning performance?
	\item H1: The perspective on the avatar has no influence on UB movements
	\item H2: The perspective on the avatar has no influence on LB movements
	\item H3: The perspective on the avatar has no influence on FB movements
	\item H4: The perspective on the avatar has no prefers no movement class, means the movement class has no influence on the learning performance 
\end{itemize}
	
\section{Outline}
After this introduction, the scope of this thesis is given. The Motor Learning movements are described as well as the classification for the Mixed Reality. In the Therory section a classification of this work in relation to the Methodology and HCI Theory. The related work part will give an overview about other and MR learning systems and also work about perspectives on avatars. From this work the measures, dependent and independent variables and tasks are derived. Taking the related work into consideration a study design is proposed in the Study Setting section. Furthermore a outlook is given in the last section.
