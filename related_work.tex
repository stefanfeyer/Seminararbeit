\chapter{Related Work}
beschr
\section{Exo-Centric}
beschr
\subsection{Task}
task
\subsection{MR tech}
mr
\subsection{tracking tech}
track
\subsection{behaviour of instruction}
behav
\subsection{measures (for dependent vars)}
measures
\subsection{considered body parts}
bp
\subsection{one paper in detail}
detail
\subsection{conclusion}
\begin{table}[]
	\begin{tabular}{|l|l|l|l|l|l|}
		\hline
		Task & MR Tech & tracking tech & instruction & measures & body parts  \\ \hline
		&  &  &  &  &  \\ \hline
	\end{tabular}
\end{table}
conclusion
\section{Ego-Centric}
beschr
\subsection{Task}
task
\subsection{MR tech}
mr
\subsection{tracking tech}
track
\subsection{behaviour of instruction}
behav
\subsection{measures (for dependent vars)}
measures
\subsection{considered body parts}
bp
\subsection{one paper in detail}
detail
\subsection{conclusion}
\begin{table}[]
	\begin{tabular}{|l|l|l|l|l|l|}
		\hline
		Task & MR Tech & tracking tech & instruction & measures & body parts  \\ \hline
		&  &  &  &  &  \\ \hline
	\end{tabular}
\end{table}
conclusion

\section{Ego-Exo Combined}
beschr
\subsection{Task}
task
\subsection{MR tech}
mr
\subsection{tracking tech}
track
\subsection{behaviour of instruction}
behav
\subsection{measures (for dependent vars)}
measures
\subsection{considered body parts}
bp
\subsection{one paper in detail}
detail
\subsection{conclusion}
\begin{table}[]
	\begin{tabular}{|l|l|l|l|l|l|}
		\hline
		Task & MR Tech & tracking tech & instruction & measures & body parts  \\ \hline
		&  &  &  &  &  \\ \hline
	\end{tabular}
\end{table}
conclusion

\section{Conclusion}
A2 dependent \markAtwoDependent \\
A5 task \markAfiveTask \\
a8 MR tech \markAeightMRtech \\
A9 track tech \markAnineTrackTech \\
A11 behav \markAelevenBehav \\
A12 measures \markAtwelveMeasures \\
A13 BP \markAonethreeBP \\
\begin{table}[]
	\begin{tabular}{|l|l|l|l|l|l|}
		\hline
		Task & MR Tech & tracking tech & instruction & measures & body parts  \\ \hline
		&  &  &  &  &  \\ \hline
	\end{tabular}
\end{table}

\begin{comment}

%----------------------------------------------------
Researchers utilised the theory of the last chapter to design MR Motor Learning systems in various ways. They differ in the technology, tasks, perspectives measures etc. In this chapter we analyse these systems to extract valuable insights to design a MR Motor learning System. First we analyse the aspects of MRML systems. After that we take a close look on some systems how they conducted their investigation and their outcome.

\section{Aspects of MR Motor Learning systems}
paper to come:
\begin{itemize}
	\item Cruz: Cyclone uppercut
	\item Davcev: AR Environment for Dance Learning
	\item Chan: Immersive Performance training Tools Using Motion Capture Technology
	\item Han: AR-Arm: Augmented Visualization for Guiding Arm Movment in the First-Person Perspective
	\item han: My Tai-Chi Coaches: An Augmented-Learning Tool for Practicing Tai-Chi Chuan
\end{itemize}

\subsection{Method}
Jacky Chan et al. \todo created a VR dance training system using an optical motion capturing system to compare the movements performed by the student with movements from the avatar. These movements are presented to the student as a 3D rendering on large screen. The movements of the students are visualised on the same screen as a coloured stick figure. The student mimics these movements and gets instant feedback as well as a feedback as a summary.

In contrast, Onebody by Hoang et al. \todo use a VR headset for a first person remote posture guidance system. 

\subsection{Tasks}
In Chan et al. \todo the dance student is presented a virtual avatar performing dance moves of A-go-go or Hip-Hop style. The avatars movement is based on the motion capturing data of a professional dancer. Onebody \todo is not only restricted to dance moves but also include other posture based sports or physical activities like Yoga or Mixed Martial Arts.

Onebody \todo uses a number of martial arts postures or stances.
\begin{itemize}
	\item Onebody: 16 artificial postures not from but like: tai chi, matial arts
	\item VR Dance Trainer: dance movements, 15 min for each move
	\item you move: various movements to perform. using a whip, baseball, boxing, ballet, dance moves
	\item training archived physical skills IVE: physical skills in sport activities, especially baseball pitching
\end{itemize}

\subsection{Measures and variables}
Jacky Chan et al. \todo defined 19 body parts that are considered in the measure of the performance of the dancing student. They name three features to compare the difference between two motions common: joint position, joint velocity and joint angle. Chan investigated which of these features suits most to judge the two dancing motions. The outcome of this investigation names the joint position to have the highest discriminative power. Hence, the joint position suits them best for their evaluation, Chan et al. calculate a score of the position error for each of the defined body parts, as well as an overall score. 

Onebody \todo uses skeletal of the instructor and the student
"Posture accuracy is determined by the extend to which the student can replicate the final posture as instructed and demonstrated as by the instructor." Independent variable: mothods for posture training. dependent variables: performance factors of posture accuracy, completion time, subjective instructor rating, users preference.

\begin{itemize}
	\item scientific work, how to measure movements: hachimura et al, yoshimura et al, qian et al, kwon et al, all use joint angles,  (mentioned in: vr dance trainer)
	\item onebody: skeletal tracking, how much percent do the postures match? 3d positions of limbs measured: wrist, elbow, shoulder, hip knee ankle. so angle between bones is the main measure for accuracy. additionally a subjective instructor score was recorded. And, completion time, topped by 2 min.
	\item VR Dance trainer: there are 3 common features for measuring hte difference between movements: joint position, velocity, angle. they tested which feature descibes movements best: joint position. base line vs post training movements are compared.
	\item you move: in each keyframe, score based on the joint with the maximum error, measured in euclidean distance. but only "important joints" are measured. timing errors: 0.5s error on each side of the frame for matching posture. is timing important, the window is reduced to 0.25s. max eucl. distance is linear mapped to a score. 0 error is 10 (max), 10cm is 7.5 what is the score to pass. if precision is important, 10cm needed for pass.
\end{itemize}

\subsection{Considered Body Parts}
scientific work, how to measure movements: hachimura et al, yoshimura et al, qian et al, kwon et al, all use joint angles

error prevention \todo

\section{Detailed description of 6-10 papers incl. Table}
%hier werden die paper detailiert vorgestellt von denen ich dann meine tasks, measures, methode und variablen ableite. am ende zusammenfassung in einer tabelle
\subsection{Onebody: Remote Posture Guidance System using First Person View in Virtual Environment}
\begin{itemize}
	\item[Hardware:] Kinect, Oculus Rift
	\item[Task:] sports, dance, martial arts, yoga
	\item[Perspectives:] First Person of the teacher
	\item[Measures:] Position, completion time, subjective score
	\item[investigation:] comparing different remote guidance systems: onebody, pre recorded video, video conference, VR third person
	\item[variables:] Independent: training method, dependent: performance
	\item[Outcome:] Onebody offers better posture accuracy in delivering movement instructions
\end{itemize}
Onebody by Hoang et al. \cite{Reinoso2016} is a VR system for remote posture guidance. Onebody is designed for sports or physical activity training like yoga, dance or martial arts. The student and the teacher are both tracked by skeletal tracking. The visualisation of this tracking are shown via a VR headset, allowing the student to follow the instruction of the teacher in first person view of the teacher - which means the student "stands inside the body of the teacher". Both, the student and the teacher are visualised by stick figures. The teachers avatar is red, the students blue and matching joints are green like shown in figure \ref{fig:ob1} left. Figure \ref{fig:ob1} right shows the scene from the first person perspective.
\begin{figure}
	\centering
	\includegraphics[width=0.225\textwidth]{img/onebody1.png}
	\includegraphics[width=0.225\textwidth]{img/onebody2.png}
	\includegraphics[width=0.45\textwidth]{img/onebody3.png}
	\caption{Left: student avatar (blue) and teacher avatar (red). Green limbs are matching limbs. Right: students view on the scene.\cite{Reinoso2016}}
	\label{fig:ob1}
\end{figure}
When the teacher moves his limbs, the student can see the movement emerging from himself. Now the student can move his own limbs to mimic the movement till the teachers posture is matched. The teacher sees the students limbs likewise allowing him to give instant feedback to the student. Thus, "Onebody provides a medium to deliver body movement instructions for non-collocated instructor and learner." \todo. The visualisations are attached to the hip but keeps the mapping between the user and corresponding avatar. To overcome different body sizes, the avatars are normalised and scaled to the size of the person seeing the avatars.\\
For transfering data, both the teacher and the student are clients in a server-client system. The clients are sending the their tracking data to the server which is broadcasting it to the clients. The comparison of the limbs for colour coding is performed on the client side. The matching of the limbs is calculated by the position of the single limbs (see equation \eqref{eq:constanterror}) with a threshold of 5cm to reduce jitter and tracking errors. Limbs in question are wrist, elbow, shoulder, hip, knee and ankle. The feedback with colour codes is provided in realtime.\\
With this system hoang et al. designed a user study to evaluate the performance of posture accuracy and user's preference. Their main hypotheses is "\textit{Onebody delivers better posture accuracy than existing remote movement instruction methods}". "Posture accuracy is determined by the extend to which the student can replicate the final posture as instructed and demonstrated by the instructor." In addition, completion time and a subjective score of the instructor are considered.
To test the hypothesis, Onebody was compared with three other remote posture training methods (independent variables): pre recorded video, video conference (Skype), VR 3rd person perspective. Each of the systems differs to Onebody in terms of synchronous interaction, VR medium and perspective see figure \ref{fig:ob2}.
\begin{figure}
	\centering
	\includegraphics[width=0.5\textwidth]{img/onebody_training_methods.PNG}
	\caption{Training methods and their differences used in the study to evaluate Onebody \cite{Reinoso2016}}
	\label{fig:ob2}
\end{figure}
The study was a 4x4 within subject. Each participant stated with a training session in which the not collocated instructor teach a posture physically and verbally. Verbal feedback was given the training repeated until the student was confident. After that the final posture was recorded. A set of four of postures with every system were performed with different complexities.\\
The results show a significant difference in accuracy. Onebody performed significantly better in over video conference, 3rd person VR and pre recorded video. Furthermore, the completion time was significantly higher with Onebody as in the other three systems. The subjective score of the instructor showed no significant differences. A post questionnaire indicated that Onebody is harder to understand and use than the other systems, but at the same time it also indicated that Onebody was perceived to be more exact. Participants rated video conference as their most preferred system over Onebody and 3rd person VR.

\subsection{Training for Physical Tasks in Virtual Environments: Tai Chi}
\begin{itemize}
	\item[Hardware:] HMD, Optical motion tracking
	\item[Task:] Tai Chi
	\item[Perspectives:] Ego-centric, exo-centric, combined
	\item[Measures:] position
	\item[investigation:] different perspectives on virtual avatar(s)
	\item[variables:] independent: perspectives, dependent: precision
	\item[Outcome:] None representation proved to be significantly better than traditional virtual Tai Chi teacher 
\end{itemize}

\subsection{A VR Dance Training System Using Motion Capture Technology}
\begin{itemize}
	\item[Hardware:] Optical Motion tracking, 3D screen
	\item[Task:] Dance moves
	\item[Perspectives:] exo-centric
	\item[Measures:] position
	\item[investigation:] Comparing video based learning with VR learning with feedback
	\item[variables:] independent: training method, dependent: precision
	\item[Outcome:] better assists in learning compared to traditional video approach, as well as more motivation an fun
\end{itemize}

\subsection{YouMove: Enhancing Movement Training with an Augmented Reality Mirror}
\begin{itemize}
	\item[Hardware:] Augmented reality mirror, Kinect
	\item[Task:] dance moves, sport moves
	\item[Perspectives:] exo-centric
	\item[Measures:] postition
	\item[investigation:] Comparing video based learning with YouMove
	\item[variables:] independent: training method, dependent: precision
	\item[Outcome:] learning and short-term retention better than traditional video representation
\end{itemize}

\subsection{Generic Heading, input? \todo}
\begin{itemize}
	\item[Hardware:] 
	\item[Task:] 
	\item[Perspectives:] 
	\item[Measures:] 
	\item[investigation:] 
	\item[variables:] 
	\item[Outcome:] 
\end{itemize}

\subsection{Generic Heading, any ideas? \todo}
\begin{itemize}
	\item[Hardware:] 
	\item[Task:] 
	\item[Perspectives:] 
	\item[Measures:] 
	\item[investigation:] 
	\item[variables:] 
	\item[Outcome:] 
\end{itemize}

\section{Research Gap}
lücken in der aktuellen forschung die sich lohnen zu untersuchen. \todo

\subsection{Conclusion of VR motor learning systems}
to conclude we have a look a Table \ref{table:ConclusionVRSystems}
\begin{table}[]
	\begin{tabular}{|l|l|l|l|l|l|}
		\hline
		Name & Hardware & Perspectives & Task & Measures & Variables \\ \hline
		Onebody &  &  &  &  &  \\ \hline
		VR dance trainer &  &  &  &  &  \\ \hline
		&  &  &  &  &  \\ \hline
		&  &  &  &  &  \\ \hline
		&  &  &  &  &  \\ \hline
		&  &  &  &  &  \\ \hline
	\end{tabular}
	\caption{Summary of proposed MR learning systems}
	\label{table:ConclusionVRSystems}
\end{table}


Research questions?\\
Hypothesis?\\
\end{comment}