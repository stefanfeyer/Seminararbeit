\chapter{Related Work}
wie haben die anderen diese variablen untersucht
wie wurden die variablen untersucht $\rightarrow$ studiensetting
%----------------------------------------------------
\section{MR learning systems}
\begin{itemize}
	\item one body:
	\item vr dance trainer:
	\item you move:
	\item training archived physical skills:
	\item
	\item teaching traditional dance using e-learning tools, bakogianni
\end{itemize}

%----------------------------------------------------
\section{Ego/exo perspective work - if exists}
\begin{itemize}
	\item training archived physical skill: seamless change from first to third person
\end{itemize}
%----------------------------------------------------
\section{variables}
\begin{itemize}
	\item independent/dependent variables
	\item measures
	\item task: reuse or adapt existing task
\end{itemize}
%----------------------------------------------------
\subsection{Task}
\begin{itemize}
	\item Onebody: 16 artificial postures not from but like: tai chi, matial arts
	\item VR Dance Trainer: dance movements, 15 min for each move
	\item you move: various movements to perform. using a whip, baseball, boxing, ballet, dance moves
	\item training archived physical skills IVE: physical skills in sport activities, especially baseball pitching
\end{itemize}
%----------------------------------------------------
\subsection{Measures}
\begin{itemize}
	\item scientific work, how to measure movements: hachimura et al, yoshimura et al, qian et al, kwon et al, all use joint angles,  (mentioned in: vr dance trainer)
	\item onebody: skeletal tracking, how much percent do the postures match? 3d positions of limbs measured: wrist, elbow, shoulder, hip knee ankle. so angle between bones is the main measure for accuracy. additionally a subjective instructor score was recorded. And, completion time, topped by 2 min.
	\item VR Dance trainer: there are 3 common features for measuring hte difference between movements: joint position, velocity, angle. they tested which feature descibes movements best: joint position. base line vs post training movements are compared.
	\item you move: in each keyframe, score based on the joint with the maximum error, measured in euclidean distance. but only "important joints" are measured. timing errors: 0.5s error on each side of the frame for matching posture. is timing important, the window is reduced to 0.25s. max eucl. distance is linear mapped to a score. 0 error is 10 (max), 10cm is 7.5 what is the score to pass. if precision is important, 10cm needed for pass.
\end{itemize}
%----------------------------------------------------
\section{Body parts included}
\begin{itemize}
	\item onebody
	\item vr dance trainer
	\item you move: full body
	\item training archived physical skill: full body
\end{itemize}
%----------------------------------------------------
\subsection{training method}
\begin{itemize}
	\item one body:
	\item vr dance trainer:
	\item you move: 
	\item training archived physical skill: key frame method
\end{itemize}
%----------------------------------------------------
\subsection{tracking technology}
\begin{itemize}
	\item 
	\item 
	\item 
	\item training archived physical skills: kinect
\end{itemize}
%----------------------------------------------------
\subsection{display technology}
\begin{itemize}
	\item
	\item 
	\item 
	\item training archived physical skills: low res HMD (oculus rift dk1/dk2)
\end{itemize}
%----------------------------------------------------
\subsection{Independent and Dependent Variables}
\subsubsection{Dependent Variables}
\begin{itemize}
	\item VR
	\item bilateral movements
	\item Movement types: synchronous  / asynchronous 
\end{itemize}
\subsubsection{Independent Variables}
\begin{itemize}
	\item Body parts: upper body (UB), lower body (LB), full body (FB)
	\item Perspective: Ego, Exo, Ego/Exo combined
	\item Movement types: synchronous  / asynchronous 
\end{itemize}

%----------------------------------------------------
\section{Conclusion}
\begin{itemize}
	\item task is xyz because of abc
	\item measures are xyz because of abc
	\item variables are...
\end{itemize}
